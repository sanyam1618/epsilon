\documentclass[11pt]{amsart}

% ------------------------------------------------------------
% Encoding and Language
% ------------------------------------------------------------
\usepackage[utf8]{inputenc}
\usepackage[english]{babel}
% \usepackage[OT2, T1]{fontenc}

% ------------------------------------------------------------
% Page Layout and Spacing
% ------------------------------------------------------------
\usepackage[a4paper,margin=1in]{geometry} % Adjust margins here
\usepackage{setspace}
\setcounter{tocdepth}{1}

% ------------------------------------------------------------
% Fonts and Symbols
% ------------------------------------------------------------
\usepackage{textcomp}
\usepackage{marvosym}
\usepackage{physics}
\usepackage{csquotes}

% ------------------------------------------------------------
% Math Packages
% ------------------------------------------------------------
\usepackage{amsmath, amsfonts, amssymb, amsthm}
\usepackage{commath}
\usepackage{tikz-cd}

% ------------------------------------------------------------
% Theorem Environments
% ------------------------------------------------------------
\usepackage{thmtools}
\newtheorem{thm}{Theorem}
\newtheorem{conj}{Conjecture}
\newtheorem{prop}[thm]{Proposition}
\newtheorem{cor}[thm]{Corollary}
\newtheorem{lem}[thm]{Lemma}
\newtheorem{conv}{Convention}
\newtheorem{fact}{Fact}
\newtheorem{defn}{Definition}
\newtheorem{que}{Question}
\newtheorem{eg}{Example}[section]
\newtheorem*{remark}{Remark}
\newtheorem*{remarks}{Remarks}
\newtheorem{note}{Note}

% ------------------------------------------------------------
% Color and Boxes
% ------------------------------------------------------------
\usepackage[dvipsnames,x11names]{xcolor} % to allow more named colors
\usepackage{tcolorbox}
\tcbuselibrary{theorems,skins,breakable}
\usepackage{framed}

% ------------------------------------------------------------
% Graphics and Layout
% ------------------------------------------------------------
\usepackage{graphicx}
\usepackage{float}
\usepackage{adjustbox}
\usepackage{fancyhdr}

% ------------------------------------------------------------
% Links and URLs
% ------------------------------------------------------------
\usepackage{hyperref}
\hypersetup{
    colorlinks=true,
    linkcolor=MidnightBlue,
    citecolor=ForestGreen,
    urlcolor=RoyalBlue
}
\usepackage{url}

% ------------------------------------------------------------
% Subfiles and Modularity
% ------------------------------------------------------------
\usepackage{subfiles}
\usepackage{environ}
\usepackage{epigraph}
% Math Operators
% ------------------------------------------------------------
\DeclareMathOperator{\Hom}{Hom}
\DeclareMathOperator{\Ext}{Ext}
\DeclareMathOperator{\Aut}{Aut}
\DeclareMathOperator{\End}{End}
\DeclareMathOperator{\Gal}{Gal}
\DeclareMathOperator{\Gl}{GL}
\DeclareMathOperator{\Sl}{SL}
\DeclareMathOperator{\SO}{SO}
\DeclareMathOperator{\GO}{GO}
\DeclareMathOperator{\id}{id}
\DeclareMathOperator{\Fr}{Frac}
\DeclareMathOperator{\Ms}{\textit{Meas}}
\DeclareMathOperator{\Cl}{Cl}
\DeclareMathOperator{\Mp}{Mp}
\DeclareMathOperator{\Ker}{Ker}
\DeclareMathOperator{\im}{Im}
\DeclareMathOperator{\ur}{ur}
\DeclareMathOperator{\Nm}{N}
\DeclareMathOperator{\Frob}{Frob}
\DeclareMathOperator{\ord}{ord}
\DeclareMathOperator{\supp}{supp}
\DeclareMathOperator{\Rep}{Rep}
\DeclareMathOperator{\rec}{rec}
\newcommand{\Upd}{\operatorname{U}^{\mathrm{M}}}

% ------------------------------------------------------------
% Common Shortcuts
% ------------------------------------------------------------
\newcommand{\N}{\mathbb{N}}
\newcommand{\h}{\mathbb{H}}
\newcommand{\A}{\mathbb{A}}
\newcommand{\I}{\mathbb{I}}
\newcommand{\R}{\mathbb{R}}
\newcommand{\Q}{\mathbb{Q}}
\newcommand{\F}{\mathbb{F}}
\newcommand{\Z}{\mathbb{Z}}
\newcommand{\C}{\mathbb{C}}
\newcommand{\p}{\mathbb{P}}

% ------------------------------------------------------------
% Document Metadata
% ------------------------------------------------------------
\newcommand{\subtitle}[1]{%
  \posttitle{\par\large#1\par\medskip}
}

\title{First talk}
\author{by Ivan}
\date{June 25, 2025}


\renewcommand\qedsymbol{$\blacksquare$}

% ------------------------------------------------------------
% Begin Document
% ------------------------------------------------------------
\begin{document}
\maketitle
\tableofcontents

\section{Locally Profinite Groups}

\begin{defn}
A topological group $G$ is called \textit{locally profinite} if it is Hausdorff, locally compact, totally disconnected and has a basis of neighborhoods of the identity consisting of open compact subgroups.
\end{defn}

Equivalently, A locally profinite group is a topological group $G$ such that
every open neighbourhood of the identity in $G$ contains a compact open subgroup of $G$. In fact, we can write
\[
G = \projlim_{K} G/K
\]
where the limit runs over all open normal subgroups $K$ of $G$.


\begin{defn}
    A local field $F$ is a field with a non-trivial absolute value $\abs{\cdot}$, that is locally compact under the topology induced by $\abs{\cdot}$ and whose topology is not discrete.   
\end{defn}


If $(F,\abs{\cdot})$ is a valued field, then we say that it is archimedean if the absolute value $\abs{\cdot}$ is archimedean, i.e., if for every $x,y \in F$ with $\abs{x} < \abs{y}$, there exists an integer $n$ such that $\abs{nx} > \abs{y}$. Otherwise, we say that $(F,\abs{\cdot})$ is non-archimedean.

We say that $F$ is a non-archimedean local field if the topology on $F$ is induced by a non-archimedean absolute value. Examples of non-archimedean local fields are finite extensions of $\Q_p$ and $\F_q((t))$. Examples of archimedean local fields are $\R$ and $\C$. 

Let $F$ be a non-Archimedean local field. Thus $F$ is the field of fractions of a discrete valuation ring $\mathcal{O}$. Let $\mathfrak{p}$ be the maximal ideal of $\mathcal{O}$ and $k= \mathcal{O}/\mathfrak{p}$ the residue class field. We will always assume that $k$ is finite, and we will generally denote the cardinality $|k|$ by $q$.


Let $\pi$ be a prime element of $F$, that is, an element satisfying
\[
\pi \mathcal{O} = \mathfrak{p},
\]
where $\mathfrak{p}$ is the maximal ideal of the ring of integers $\mathcal{O}$ of $F$.
Every element $x \in F^\times$ admits a unique factorization
\[
x = u \pi^n,
\]
with $u \in \mathcal{O}^\times = U_F$ a unit and $n \in \mathbb{Z}$.
We write $n = \nu_F(x)$ for the normalized valuation of $x$.

The field $F$ carries the absolute value
\[
\lVert x \rVert = q^{-n} = q^{-\nu_F(x)},
\]
where $q = \lvert \mathcal{O}/\mathfrak{p} \rvert$.
We set $\lVert 0 \rVert = 0$, so that $\lVert x \rVert > 0$ for $x \neq 0$.
This absolute value defines a metric under which $F$ is complete, making $F$ a topological field.

For each $n \in \mathbb{Z}$, the fractional ideal
\[
\mathfrak{p}^n
   = \pi^n \mathcal{O}
   = \{x \in F : \lVert x \rVert \le q^{-n}\}
\]
is an open additive subgroup of $F$, and the collection $\{\mathfrak{p}^n\}_{n \in \mathbb{Z}}$
forms a fundamental system of neighborhoods of $0$.

Because $F$ is complete and the residue field $k = \mathcal{O}/\mathfrak{p}$ is finite,
the canonical map
\[
\mathcal{O} \;\longrightarrow\; \varprojlim_{n} \mathcal{O}/\mathfrak{p}^n
\]
is a topological isomorphism.
Each quotient $\mathcal{O}/\mathfrak{p}^n$ is finite, hence the inverse limit is compact.
Moreover, each fractional ideal $\mathfrak{p}^n$ is topologically isomorphic to $\mathcal{O}$
and is therefore compact.
It follows that the additive group $(F,+)$ is \emph{locally profinite},
and $F$ is the union of its compact open subgroups.

The same reasoning shows that the multiplicative group $F^\times$ is locally profinite.
Standard arguments then imply that for every $n \ge 1$, the groups
\[
F^n,\quad M_n(F),\quad \mathrm{GL}_n(F),\quad
\mathrm{SL}_n(F),\quad
\mathrm{SO}_n(F),\quad
\mathrm{GO}_n(F),\quad
\mathrm{Mp}_n(F)
\]
are all locally profinite as well.


\section{Characters of Locally Profinite Groups}

\begin{defn}
    Let $G$ be a locally profinite group. A character of $G$ is a continuous homomorphism $\chi: G \to \mathbb{C}^\times$. We say that a character is unitary if its image lies in the unit circle $S^1 \subset \mathbb{C}^\times$.
\end{defn}

For a local field $F$ we write $\hat{F}$ for the group of unitary characters of the additive group $(F,+)$. 


\cite{BH06}




\nocite{}
\bibliographystyle{amsplain}
\bibliography{references.bib}

\end{document}
